\documentclass[a4paper,11pt]{article}

\usepackage{amsmath}
\usepackage{amssymb}

\title{An representation of structured submeshes for code generation}
\begin{document}

\maketitle

\section{Idea}
\label{sec:idea}

We are currently applying ad-hoc code generation to exploit structure
of various kinds in meshes. Be that regular ``block-structured''
refinement of unstructured meshes, or columnar extrusion of
unstructured meshes. Ideally, we would like to be able to do both.

Here, we describe a symbolic representation of the sub-structuring patterns
and iteration over the structure that should enable code generation
without hand-coding all of the individual cases.

We will represent the structuring operation by an object
\texttt{Structure}. Conceptually, we think of the structuring as
operating on each element $e_o$ in a mesh $M_o$ and producing, for
each element, a new mesh $M_n$with some number of new elements $e_n$
(which may be of a different type to the original mesh).

For the purposes of code generation, we assume that we produce
code that operates on each $e_o$ in turn, and is fed data in a
canonical ``reference element'' order. It is the job of the outside
iteration engine to ensure that this is the case.

\section{Describing structure}
\label{sec:describing-structure}

We will want to index different entity types in the structured mesh
separately (for example, iterating over all cells, or over all faces).
Call these \emph{classes} of entities, with each class containing a
single type of entity. We label each entity in a class by an integer
$i_e \in [0, N_e)$, with $N_e$ the number of entities of class $e$. To
provide an iteration order over the entities, we will use a polyhedral
set $\Phi_e$. We index each entity by a multiindex
$\alpha_e \in \Phi_e$. For example, for refinement of a quad into 16
quads, we can describe the cells with
\begin{equation}
  \Phi_{\text{cell}} = \{(i, j) \colon 0 \le i < 4, 0 \le j < 4 \}.
\end{equation}
An iteration order is obtained through a map
\begin{equation}
  \label{eq:3}
  I_e \colon \Phi_e \to [0, N_e).
\end{equation}
For example, lexicographic iteration over $\Phi_{\text{cell}}$ would
be obtained by choosing
\begin{equation}
  \label{eq:4}
  I_\text{cell}(i, j) = 4*j + i.
\end{equation}
A tiling of the 2D domain can be achieved by splitting the indices of
$\Phi_{\text{cell}}$ and an appropriate loop order (imposed by choice
of $I_\text{cell}$).

These iteration indices can be used to compute offsets into the local
vector by an affine transformation.

In addition to mapping from points in the set to iteration indices
(and hence dofs), we also need some topological queries. For example,
given a cell multiindex and an index of a subentity of the cell, we
should be able to return the appropriate subentity multiindex. We also
need the reverse (multi-)map. That is, denoting a subentity on the
reference element by the multiindex $\beta$, we need a map
\begin{equation}
  \label{eq:5}
  T_{e,k} : \Phi_e \times \beta \to \Phi_k.
\end{equation}
Similarly we need the dual operation mapping
\begin{equation}
  \label{eq:6}
  T^*_{k,e} : \Phi_k \to \Phi_e \times \beta.
\end{equation}
We only need to explicitly provide mappings between entities of
neighbouring dimension, since the remainder can be obtained by composition.
Note, in DMPlex-like terminology, these operations are called \texttt{cone}
and \texttt{support} respectively.

\section{Example: extruded meshes}
\label{sec:exampl-extr-mesh}

\subsection{Entity sets}
\label{sec:entity-sets}

Extruding a single triangular cell gives us $N_c$ prismatic cells in a
column. Enumerating the entity sets, using the notation $(a, b)$ for
an entity with co-dimension $a$ on the base cell and co-dimension $b$
on the extruding interval, we have cells:
\begin{equation}
  \label{eq:7}
  \Phi_{(0, 0)} = \{i \colon 0 \le i < N_c\},
\end{equation}
horizontal faces
\begin{equation}
  \label{eq:8}
  \Phi_{(0, 1)} = \{i \colon 0 \le i < N_c + 1\},
\end{equation}
vertical faces
\begin{equation}
  \Phi_{(1, 0)} = \{(i,j) \colon 0 \le i < N_c, 0 \le j < 3\},
\end{equation}
horizontal edges
\begin{equation}
  \label{eq:9}
  \Phi_{(1, 1)} = \{(i,j) \colon 0 \le i < N_c+ 1, 0 \le j < 3\},
\end{equation}
vertical edges
\begin{equation}
  \label{eq:10}
  \Phi_{(2, 0)} = \{(i, j) \colon 0 \le i < N_c, 0 \le j < 3\},
\end{equation}
and vertices
\begin{equation}
  \label{eq:11}
  \Phi_{(2, 1)} = \{(i, j) \colon 0 \le i < N_c +1, 0 \le j < 3\}.
\end{equation}

\subsection{``Cone'' maps}
\label{sec:cone-maps}

Now the forward maps.
Cells to horizontal faces
\begin{equation}
  \label{eq:1}
  T_{(0, 0), (0, 1)}(i, \beta) =
  \begin{cases}
    i & \beta = 0 \text{ bottom face}\\
    i+1 & \beta = 1 \text{ top face}
  \end{cases},
\end{equation}
to vertical faces
\begin{equation}
  \label{eq:2}
  T_{(0, 0), (1, 0)}(i, \beta) = (i, \beta) \quad \text{ $\beta \in \{0, 1, 2\}$}.
\end{equation}
Horizontal faces to horizontal edges
\begin{equation}
  \label{eq:12}
  T_{(0,1),(1,1)}(i,\beta) = (i, \beta) \quad \text{ $\beta \in \{0, 1, 2\}$},
\end{equation}
vertical faces to horizontal edges
\begin{equation}
  \label{eq:13}
  T_{(1,0),(1,1)}((i, j),\beta) =
  \begin{cases}
    (i, j) & \beta = 0 \text{ bottom edge}\\
    (i+1, j) & \beta = 1 \text{ top edge}
  \end{cases},
\end{equation}
vertical faces to vertical edges
[FIXME: This supposes a particular numbering convention of vertices and
edges on a triangle].
\begin{equation}
  \label{eq:14}
  T_{(1,0),(2,0)}((i, j),\beta) =
  \begin{cases}
    (i, j) & \beta = 0 \text{ right edge}\\
    (i, j + 1 \mod 3) & \beta = 1 \text{ left edge}
  \end{cases}.
\end{equation}

Now the maps from edges to vertices. First horizontal edges to
vertices
\begin{equation}
  \label{eq:15}
  T_{(1,1),(2,1)}((i,j), \beta) =
  \begin{cases}
    (i, j) & \beta = 0 \text{ right vertex}\\
    (i, j + 1 \mod 3) & \beta = 1 \text{ left vertex}
  \end{cases},
\end{equation}
and vertical edges to vertices
\begin{equation}
  \label{eq:16}
  T_{(2, 0),(2, 1)}((i, j), \beta) =
  \begin{cases}
    (i, j) & \beta = 0 \text{ bottom vertex}\\
    (i+1, j) & \beta = 1 \text{ top vertex}
  \end{cases}.
\end{equation}

\subsection{``Support'' maps}
\label{sec:support-maps}
These maps are dual to the cone maps above. Again, we only need to
explicitly specify relations between entities of neighbouring
codimension, since the remainder can be built by appropriate
composition.

Rather than returning a single (indexed) entity here, we return a
subset of the pointed-to set, with the bounds on the indices
implicitly filtering invalid entries.

Vertex to vertical edges
\begin{equation}
  \label{eq:17}
  T^*_{(2,1),(2,0)}(i, j) = \{(i-1, j), (i, j)\},
\end{equation}
vertex to horizontal edges
\begin{equation}
  \label{eq:18}
  T^*_{(2,1), (1, 1)}(i, j) = \{(i, j-1 \mod 3), (i, j)\},
\end{equation}
vertical edge to vertical faces
\begin{equation}
  \label{eq:19}
  T^*_{(2, 0), (1, 0)}(i, j) = \{(i, j - 1 \mod 3), (i, j)\},
\end{equation}
horizontal edge to vertical faces
\begin{equation}
  \label{eq:20}
  T^*_{(1, 1), (1, 0)}(i, j) = \{(i - 1, j), (i, j)\},
\end{equation}
horizontal edge to horizontal faces
\begin{equation}
  \label{eq:20}
  T^*_{(1, 1), (0, 1)}(i, j) = \{i\},
\end{equation}
vertical face to cells
\begin{equation}
  \label{eq:21}
  T^*_{(1, 0), (0, 0)}(i, j) = \{i\},
\end{equation}
and horizontal face to cells
\begin{equation}
  \label{eq:22}
  T^*_{(0, 1), (0, 0)}(i, j) = \{i-1, i\}.
\end{equation}
\end{document}
